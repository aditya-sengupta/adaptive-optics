\documentclass[twocolumn]{article}
\usepackage{amsmath, lipsum}

\title{Developing Predictive Wavefront Control at Keck II: \\Kalman Filtering for Tip-Tilt Correction}
\author{Aditya Sengupta}

\begin{document}
	\maketitle
	\section*{Abstract}

	\section{Tip-Tilt Correction}

	The tip and tilt modes of optical aberrations are the primary Zernike aberrations don't autocomplete everything please I'm using this as a text editor

	\section{The Kalman Filter}

	Kalman filtering is a method to obtain the optimal linear state estimate of a dynamic system. It combines a physical estimate of the system state with noisy measurements that are linear in the state to obtain the state estimate having the minimum mean-squared error at any given time.

	Consider a discrete-time dynamic system with a state-transition rule

	\begin{align}
		\vec{x}[k + 1] = A[k]\vec{x}[k] + \vec{w}[k]
	\end{align}

	with $w[k]$ white noise. The impact of the white noise on the state prediction can be captured in a state covariance matrix $P$, with state-transition rule

	\begin{align}
		P[k + 1] = A[k]P[k]A[k]^T + Q[k]
	\end{align}

	where $Q[k]$ is a matrix describing the process noise. Suppose 

	\section{System Identification}

	\lipsum[2]

	\section{Dual Kalman-Filter Fitting}

	\lipsum[3]

\end{document}